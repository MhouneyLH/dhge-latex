% file to include additional commands, packages or anything else you'd like to include in the preamble

% Define listing styles
\usepackage{listings} % z. B. für Codebeispiele
\usepackage{xcolor} % Definition von Farben

\definecolor{greencomments}{rgb}{0,0.5,0}
\definecolor{graylinenumbers}{rgb}{0.5,0.5,0.5}
\definecolor{background}{rgb}{0.95,0.95,0.92}

% language specific styles
% \lstdefinelanguage{your_custom_language} { ... }

% Set the basic listing style for all languages
\lstset{
    basicstyle=\ttfamily\footnotesize,
    backgroundcolor=\color{background},
    commentstyle=\color{greencomments},
    numberstyle=\tiny\color{graylinenumbers},
    escapeinside={(*@}{@*)}, % Allows LaTeX commands within code: (*@ here is your commented out code then @*)
    breakatwhitespace=false,
    breaklines=true,
    captionpos=b,
    keepspaces=true,
    numbers=left,
    numbersep=5pt,
    showspaces=false,
    showstringspaces=false,
    showtabs=false,
    tabsize=2,
    frame=single,
    framexleftmargin=15pt,
    xleftmargin=15pt,
}

\usepackage{pgfplots} % Diagramme
\usepackage{multirow} % bessere Tabellen
\usepackage{array} % Enhances the array and tabular environments for creating tables.
\usepackage[toc,page]{appendix}
\usepackage{appendix}
\usepackage{amsmath} % for central dots and svg figures
\usepackage[inkscapeformat=png]{svg} % showing svgs -> see dhgesvgfigure

% sometimes just use biber.exe "template" again to update the bibliography -> sometimes it just does not load the correct entries after modifying this file :(
\usepackage{longtable} % for tables that span multiple pages
\usepackage[bottom]{footmisc} % for fixed footnotes at the bottom of the page
\usepackage{parskip} % for no indent and space between paragraphs
