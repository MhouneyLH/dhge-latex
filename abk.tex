% Definieren Sie hier Ihre Abkürzungen und Glossar-Einträge anhand der Beispiele.
% Wenn Sie diese dann im Text verwenden, rufen Sie einfach \gls{key} auf, z.B. \gls{ac:dhge}.
% LaTeX kümmert sich um den Rest.
% Wenn alle Abkürzungen auch ohne Verweis darauf generiert werden sollen, ist ein Schalter dafür in config.tex verfügbar.
% Eine ausführliche, anfängerfreundliche Dokumentation ist unter https://www.overleaf.com/learn/latex/Glossaries abrufbar.

\newglossaryentry{gls:gloss}{
    name={Glossar},
    description={Ein Glossar ist eine Liste von Wörtern mit beigefügten Bedeutungserklärungen oder Übersetzungen. (Wikipedia)}
}

\newacronym[]{ac:dhge}{DHGE}{Duale Hochschule Gera-Eisenach}
