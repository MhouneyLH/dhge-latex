% ROOT & PACKAGE SETUP
\documentclass[a4paper, 12pt]{article}

% Präambel laden

% file to include additional commands, packages or anything else you'd like to include in the preamble

% Define listing styles
\usepackage{listings} % z. B. für Codebeispiele
\usepackage{xcolor} % Definition von Farben

\definecolor{codegreen}{rgb}{0,0.6,0}
\definecolor{codegray}{rgb}{0.5,0.5,0.5}
\definecolor{codepurple}{rgb}{0.58,0,0.82}
\definecolor{backcolor}{rgb}{0.95,0.95,0.92}

\lstdefinestyle{coding_style}{
    backgroundcolor=\color{backcolor},
    commentstyle=\color{codegreen},
    keywordstyle=\color{blue},
    numberstyle=\tiny\color{codegray},
    stringstyle=\color{codepurple},
    basicstyle=\ttfamily\footnotesize,
    breakatwhitespace=false,
    breaklines=true,
    captionpos=b,
    keepspaces=true,
    numbers=left,
    numbersep=5pt,
    showspaces=false,
    showstringspaces=false,
    showtabs=false,
    tabsize=2,
    frame=single,
    framexleftmargin=15pt,
    xleftmargin=15pt
}

% Set the listing style
\lstset{style=coding_style}

\usepackage{pgfplots} % Diagramme
\usepackage{multirow} % bessere Tabellen
\usepackage{array} % Enhances the array and tabular environments for creating tables.
\usepackage[toc,page]{appendix}
\usepackage{appendix}
\usepackage{amsmath} % for central dots


% Nutzerkonfiguration laden
\def\CARBEIT        {1}
\def\CTITLE		    {Test}
\def\CAUTHORANR     {0}
\def\CAUTHORVOR		{Nutzer}
\def\CAUTHORNACH    {Test}
\def\CDATUM		    {\today}
\def\CMATRIKEL		{G42069TE}
\def\CKURS		    {TE}
\def\CCAMPUS		{Gera}
\def\CBEREICH		{Testen}
\def\CSTUDIENGANG	{TEA}
\def\CBETRIEB		{Tester  GmbH.}
\def\CBETREUER		{Dr. Eng. Tester}
\def\CSPERRVERMERK  {1}
\def\CABKL          {DHGE}
\def\CKUSCHE         {0}
\def\CHASABSTRACT    {0}
\def\CFANCYFONTS    {1}


% verwendete Pakete laden und konfigurieren
\usepackage		[a4paper,
    	            inner  = 3cm,
                    outer  = 2cm,
                    top    = 2.5cm,
                    bottom = 2.5cm]{geometry}
\usepackage		{setspace}
\usepackage		[hyperfootnotes = false,
                    hidelinks]{hyperref}
\usepackage		{amssymb}
\usepackage		{fancyhdr}
\usepackage		[version = 3]{acro}
\usepackage		{enumitem}
\usepackage		[style=german]{csquotes}
\usepackage		[backend=biber,
                    style        = alphabetic,
                    citestyle    = components/alphabetic-ibid,
                    giveninits   = true,
                    ibidtracker  = true,
                    minbibnames  = 3,
                    minalphanames= 3]{biblatex}
\usepackage		[ngerman]{babel}
\usepackage		{footmisc}
\usepackage		{graphicx}
\usepackage		{caption}
\usepackage		{xparse}
\usepackage		{float}
\usepackage		{tocloft}
\usepackage     {lmodern}
\usepackage     {totcount}
\usepackage     {chngcntr}
\usepackage     {icomma} % korrekte Darstellung von Kommas in Formeln

% schönere Fonts, aber optional. Zum deaktivieren CFANCYFONTS in config.tex auf 0 setzen
\if\CFANCYFONTS 1
\usepackage     [scaled=0.88]{beraserif} % Bera Serifen Font
\usepackage     [scaled=0.85]{berasans} % Bera Sans Font
\usepackage     [scaled=0.84]{beramono} % Bera Mono Font
\usepackage     [T1]{fontenc}
\usepackage     {mathpazo} % Palatino Font
\usepackage     [T1,small,euler-digits]{eulervm} % Euler Font
\DeclareMathAlphabet{\mathrm}{OT1}{cmr}{m}{n} % mathrm soll weiter Computer Modern als Font nutzen, siehe #103
\DeclareMathAlphabet{\mathit}{OT1}{cmr}{m}{it}
\usepackage     {listings}
\lstset{
	basicstyle=\ttfamily,
    breaklines=true
}
\fi

% VERALTETE PAKETE
% nicht mehr benötigt, aber für Rückwärtskompatibilität noch enthalten
% einkommentieren, wenn nach einem Template-Update Probleme auftauchen

%\usepackage		[utf8]{inputenc}
%\usepackage		{titletoc}
%\usepackage		{csquotes,xpatch}
%\usepackage		{ifthen}
%\usepackage		{etoolbox}


% Setup von Commands und Dokument
\DeclareDocumentCommand{\dhgefigure}{O{tbp} m O{width=\textwidth, angle=0} m m O{} O{}}
{
    \begin{figure}[#1]
        \begin{center}
            \includegraphics[#3]{#2}
        \end{center}
        \caption{#4}
        \label{#5}

        \ifx #6\empty \else
            \ifx #7\empty
                {\small \protect \textbf{Quelle:} \cite{#6}}
                \else
                {\small \protect \textbf{Quelle:} \cite[#7]{#6}}
            \fi
        \fi

    \end{figure}
}

\DeclareDocumentCommand{\dhgesvgfigure}{O{tbp} m O{width=\textwidth, angle=0} m m O{} O{}}
{
    \begin{figure}[#1]
        \begin{center}
            \includesvg[#3]{#2}
        \end{center}
        \caption{#4}
        \label{#5}

        \ifx #6\empty \else
            \ifx #7\empty
                {\small \protect \textbf{Quelle:} \cite{#6}}
                \else
                {\small \protect \textbf{Quelle:} \cite[#7]{#6}}
            \fi
        \fi

    \end{figure}
}

% SubSubSubSection
\newcommand{\dhgeparagraph}[1]{\paragraph{#1}\mbox{}\\}

% Doppelte Unterstreichung
\newcommand{\doubleunderline}[1]{
    \underline{\underline{#1}}
}

% Formatierung der Bachelorarbeit: Autorreferat und Thesenblatt
\newcommand{\baFormat}[2]{
    \begin{center}
        {\LARGE\bf #1}

        \vspace{0.7cm}
        {\large\bf\enquote{\CTITLE}}

        \vspace{0.5cm}
        von \CAUTHOR
    \end{center}

    \vspace{1.5cm}

    {#2}

    \cleardoublepage
}

% => Das LaTeX Arbeitsverzeichnis ist das der Haupt-Datei, also build/template.tex => build/

% DOCUMENT SETUP
\onehalfspacing
\widowpenalty10000
\clubpenalty10000

% INHALTSVERZEICHNIS SETUP
\cftsetindents	{section}{0em}{4em}
\cftsetindents	{subsection}{0em}{4em}
\cftsetindents	{subsubsection}{0em}{4em}
\renewcommand	{\contentsname}{Inhaltsverzeichnis}
\setcounter		{tocdepth}{3}
\setcounter		{secnumdepth}{5}

% ABBILDUNGEN UND TABELLEN SETUP

% im Kusche-Mode sollen Abbildungen nach kapitel.lfd nummeriert werden
\if\CKUSCHE 1
    \counterwithin{figure}{section}
    \counterwithin{table}{section}
\fi

\renewcommand	{\listfigurename}{Abbildungsverzeichnis}
\renewcommand	{\listtablename}{Tabellenverzeichnis}

\addto{\captionsngerman}{
    \renewcommand*{\figurename}{Abb.}
    \renewcommand*{\tablename}{Tab.}
}

\addtocontents{lof}{\linespread{2}\selectfont}
\addtocontents{lot}{\linespread{2}\selectfont}

\makeatletter
\renewcommand{\cftfigpresnum}{Abb. }
\renewcommand{\cfttabpresnum}{Tab. }

\setlength{\cftfignumwidth}{2cm}
\setlength{\cfttabnumwidth}{2cm}

\setlength{\cftfigindent}{0cm}
\setlength{\cfttabindent}{0cm}
\makeatother

% CAPTION SETUP
\captionsetup{
    font=small,
    labelfont=bf,
    singlelinecheck=false,
    skip=10pt,
    belowskip=0pt
}

\renewcommand*{\labelalphaothers}{\textsuperscript{}}

% HEADERS & FOOTERS
\pagestyle		{fancyplain}
\fancyhf		{}
\renewcommand	{\headrulewidth}{0pt}
\renewcommand	{\footrulewidth}{0pt}
\setlength		{\headheight}{15pt}

\fancyfoot      [R]{\thepage} % nach den neuen Anforderungen sind Seitenzahlen unten rechts, geht d'accord mit dem Kusche-Mode
\if\CKUSCHE 1
    \fancyfoot      [L]{\leftmark} % im Kusche-Mode erscheint linksbündig das Kapitel in der Fußzeile
\fi

% PARAGRAPH SETUP
\newcommand{\dhgeparagraph}[1]{\paragraph{#1}\mbox{}\\\vspace{-1.5em}}

% FOOTNOTE
\renewcommand{\footnotelayout}{\hspace{0.5em}}

% Conditionals um AbbildungsVZ und TabellenVZ nur zu rendern, wenn sie nicht leer sind
\newtotcounter{figCount}
\newtotcounter{tabCount}
\let\oldTabTOC=\table
\let\oldFigTOC=\figure
\renewcommand{\figure}{\stepcounter{figCount}\oldFigTOC}
\renewcommand{\table}{\stepcounter{tabCount}\oldTabTOC}

\newcommand{\conditionalLoF}{
    \ifnum\totvalue{figCount}>0
        \addcontentsline{toc}{section}{\listfigurename}
        \listoffigures
        \cleardoublepage
    \fi
}
\newcommand{\conditionalLoT}{
    \ifnum\totvalue{tabCount}>0
        \addcontentsline{toc}{section}{\listtablename}
        \listoftables
        \cleardoublepage
    \fi
}

% COUNTER (Zweck: in römischen Zahlen weiterzählen, nachdem der Counter von arabisch zurückgeändert wird)
\newcounter{savepage}

% Sections sollen mit Seitenumbruch beginnen
\let\stdsection\section
\renewcommand\section{\newpage\stdsection}

% Überschreiben von \mathrm{} -> einheitlichen Abstand einfügen
\let\oldMathrm\mathrm
\renewcommand{\mathrm}[1]{\,\oldMathrm{#1}}

% Template-Root-Verzeichnis ist das neue Arbeitsverzeichnis
% muss nach % DOCUMENT SETUP
\onehalfspacing
\widowpenalty10000
\clubpenalty10000

% INHALTSVERZEICHNIS SETUP
\cftsetindents	{section}{0em}{4em}
\cftsetindents	{subsection}{0em}{4em}
\cftsetindents	{subsubsection}{0em}{4em}
\renewcommand	{\contentsname}{Inhaltsverzeichnis}
\setcounter		{tocdepth}{3}
\setcounter		{secnumdepth}{5}

% ABBILDUNGEN UND TABELLEN SETUP

% im Kusche-Mode sollen Abbildungen nach kapitel.lfd nummeriert werden
\if\CKUSCHE 1
    \counterwithin{figure}{section}
    \counterwithin{table}{section}
\fi

\renewcommand	{\listfigurename}{Abbildungsverzeichnis}
\renewcommand	{\listtablename}{Tabellenverzeichnis}

\addto{\captionsngerman}{
    \renewcommand*{\figurename}{Abb.}
    \renewcommand*{\tablename}{Tab.}
}

\addtocontents{lof}{\linespread{2}\selectfont}
\addtocontents{lot}{\linespread{2}\selectfont}

\makeatletter
\renewcommand{\cftfigpresnum}{Abb. }
\renewcommand{\cfttabpresnum}{Tab. }

\setlength{\cftfignumwidth}{2cm}
\setlength{\cfttabnumwidth}{2cm}

\setlength{\cftfigindent}{0cm}
\setlength{\cfttabindent}{0cm}
\makeatother

% CAPTION SETUP
\captionsetup{
    font=small,
    labelfont=bf,
    singlelinecheck=false,
    skip=10pt,
    belowskip=0pt
}

\renewcommand*{\labelalphaothers}{\textsuperscript{}}

% HEADERS & FOOTERS
\pagestyle		{fancyplain}
\fancyhf		{}
\renewcommand	{\headrulewidth}{0pt}
\renewcommand	{\footrulewidth}{0pt}
\setlength		{\headheight}{15pt}

\fancyfoot      [R]{\thepage} % nach den neuen Anforderungen sind Seitenzahlen unten rechts, geht d'accord mit dem Kusche-Mode
\if\CKUSCHE 1
    \fancyfoot      [L]{\leftmark} % im Kusche-Mode erscheint linksbündig das Kapitel in der Fußzeile
\fi

% PARAGRAPH SETUP
\newcommand{\dhgeparagraph}[1]{\paragraph{#1}\mbox{}\\\vspace{-1.5em}}

% FOOTNOTE
\renewcommand{\footnotelayout}{\hspace{0.5em}}

% Conditionals um AbbildungsVZ und TabellenVZ nur zu rendern, wenn sie nicht leer sind
\newtotcounter{figCount}
\newtotcounter{tabCount}
\let\oldTabTOC=\table
\let\oldFigTOC=\figure
\renewcommand{\figure}{\stepcounter{figCount}\oldFigTOC}
\renewcommand{\table}{\stepcounter{tabCount}\oldTabTOC}

\newcommand{\conditionalLoF}{
    \ifnum\totvalue{figCount}>0
        \addcontentsline{toc}{section}{\listfigurename}
        \listoffigures
        \cleardoublepage
    \fi
}
\newcommand{\conditionalLoT}{
    \ifnum\totvalue{tabCount}>0
        \addcontentsline{toc}{section}{\listtablename}
        \listoftables
        \cleardoublepage
    \fi
}

% COUNTER (Zweck: in römischen Zahlen weiterzählen, nachdem der Counter von arabisch zurückgeändert wird)
\newcounter{savepage}

% Sections sollen mit Seitenumbruch beginnen
\let\stdsection\section
\renewcommand\section{\newpage\stdsection}

% Überschreiben von \mathrm{} -> einheitlichen Abstand einfügen
\let\oldMathrm\mathrm
\renewcommand{\mathrm}[1]{\,\oldMathrm{#1}}

% Template-Root-Verzeichnis ist das neue Arbeitsverzeichnis
% muss nach % DOCUMENT SETUP
\onehalfspacing
\widowpenalty10000
\clubpenalty10000

% INHALTSVERZEICHNIS SETUP
\cftsetindents	{section}{0em}{4em}
\cftsetindents	{subsection}{0em}{4em}
\cftsetindents	{subsubsection}{0em}{4em}
\renewcommand	{\contentsname}{Inhaltsverzeichnis}
\setcounter		{tocdepth}{3}
\setcounter		{secnumdepth}{5}

% ABBILDUNGEN UND TABELLEN SETUP

% im Kusche-Mode sollen Abbildungen nach kapitel.lfd nummeriert werden
\if\CKUSCHE 1
    \counterwithin{figure}{section}
    \counterwithin{table}{section}
\fi

\renewcommand	{\listfigurename}{Abbildungsverzeichnis}
\renewcommand	{\listtablename}{Tabellenverzeichnis}

\addto{\captionsngerman}{
    \renewcommand*{\figurename}{Abb.}
    \renewcommand*{\tablename}{Tab.}
}

\addtocontents{lof}{\linespread{2}\selectfont}
\addtocontents{lot}{\linespread{2}\selectfont}

\makeatletter
\renewcommand{\cftfigpresnum}{Abb. }
\renewcommand{\cfttabpresnum}{Tab. }

\setlength{\cftfignumwidth}{2cm}
\setlength{\cfttabnumwidth}{2cm}

\setlength{\cftfigindent}{0cm}
\setlength{\cfttabindent}{0cm}
\makeatother

% CAPTION SETUP
\captionsetup{
    font=small,
    labelfont=bf,
    singlelinecheck=false,
    skip=10pt,
    belowskip=0pt
}

\renewcommand*{\labelalphaothers}{\textsuperscript{}}

% HEADERS & FOOTERS
\pagestyle		{fancyplain}
\fancyhf		{}
\renewcommand	{\headrulewidth}{0pt}
\renewcommand	{\footrulewidth}{0pt}
\setlength		{\headheight}{15pt}

\fancyfoot      [R]{\thepage} % nach den neuen Anforderungen sind Seitenzahlen unten rechts, geht d'accord mit dem Kusche-Mode
\if\CKUSCHE 1
    \fancyfoot      [L]{\leftmark} % im Kusche-Mode erscheint linksbündig das Kapitel in der Fußzeile
\fi

% PARAGRAPH SETUP
\newcommand{\dhgeparagraph}[1]{\paragraph{#1}\mbox{}\\\vspace{-1.5em}}

% FOOTNOTE
\renewcommand{\footnotelayout}{\hspace{0.5em}}

% Conditionals um AbbildungsVZ und TabellenVZ nur zu rendern, wenn sie nicht leer sind
\newtotcounter{figCount}
\newtotcounter{tabCount}
\let\oldTabTOC=\table
\let\oldFigTOC=\figure
\renewcommand{\figure}{\stepcounter{figCount}\oldFigTOC}
\renewcommand{\table}{\stepcounter{tabCount}\oldTabTOC}

\newcommand{\conditionalLoF}{
    \ifnum\totvalue{figCount}>0
        \addcontentsline{toc}{section}{\listfigurename}
        \listoffigures
        \cleardoublepage
    \fi
}
\newcommand{\conditionalLoT}{
    \ifnum\totvalue{tabCount}>0
        \addcontentsline{toc}{section}{\listtablename}
        \listoftables
        \cleardoublepage
    \fi
}

% COUNTER (Zweck: in römischen Zahlen weiterzählen, nachdem der Counter von arabisch zurückgeändert wird)
\newcounter{savepage}

% Sections sollen mit Seitenumbruch beginnen
\let\stdsection\section
\renewcommand\section{\newpage\stdsection}

% Überschreiben von \mathrm{} -> einheitlichen Abstand einfügen
\let\oldMathrm\mathrm
\renewcommand{\mathrm}[1]{\,\oldMathrm{#1}}

% Template-Root-Verzeichnis ist das neue Arbeitsverzeichnis
% muss nach \input{setup.tex} aufgerufen werden, damit alle Template-Root-Dateien integriert werden
\makeatletter
\def\input@path{{../}{path1/}}
\makeatother

% die veraltete CAUTHOR Variable wird automatisch befüllt
\def\CAUTHOR{\CAUTHORVOR\ \CAUTHORNACH}

\newcounter{totalbibentries}
\newcommand*{\listcounted}{}

\makeatletter
\AtDataInput{
  \xifinlist{\abx@field@entrykey}\listcounted
    {}
    {\stepcounter{totalbibentries}
     \listxadd\listcounted{\abx@field@entrykey}}
}
\makeatother

% Standard Abkürzungsverzeichnis überschreiben -> einheitliche Einrückung
\RenewAcroTemplate[list]{description}{%
    \acronymsmapT{%
        \AcroAddRow{%
        \textbf{\acrowrite{short}}%
        &
        \acrowrite{long}%
        \acropages
            {\acrotranslate{page}\nobreakspace}%
            {\acrotranslate{pages}\nobreakspace}%
        \vspace{10pt}
        \tabularnewline
        }%
    }%
    \acroheading
    \acropreamble
    \noindent
    \begin{tabular}{@{}ll}
        \AcronymTable
    \end{tabular}
}

% Abkürzungsverzeichnis SETUP
\acsetup{
	list/heading = section*,
	list/name = {Abkürzungsverzeichnis},
    list/template = description,
	make-links = true,
	link-only-first = false
}

% Abkürzungsverzeichnis überschreibt \UseAcroTemplate für \ac
% New Counter to count used acronyms:
\newtotcounter{acro_num}
\def\oldUseAcroTemplate{} \let\oldUseAcroTemplate=\UseAcroTemplate
\def\UseAcroTemplate{\stepcounter{acro_num}\oldUseAcroTemplate}

% Abstände und Einrückungen abhängig von config.tex ein-/ausschalten
\if\CEINR 0
    \setlength{\parskip}{6pt}
    \setlength{\parindent}{0cm}
\fi
 aufgerufen werden, damit alle Template-Root-Dateien integriert werden
\makeatletter
\def\input@path{{../}{path1/}}
\makeatother

% die veraltete CAUTHOR Variable wird automatisch befüllt
\def\CAUTHOR{\CAUTHORVOR\ \CAUTHORNACH}

\newcounter{totalbibentries}
\newcommand*{\listcounted}{}

\makeatletter
\AtDataInput{
  \xifinlist{\abx@field@entrykey}\listcounted
    {}
    {\stepcounter{totalbibentries}
     \listxadd\listcounted{\abx@field@entrykey}}
}
\makeatother

% Standard Abkürzungsverzeichnis überschreiben -> einheitliche Einrückung
\RenewAcroTemplate[list]{description}{%
    \acronymsmapT{%
        \AcroAddRow{%
        \textbf{\acrowrite{short}}%
        &
        \acrowrite{long}%
        \acropages
            {\acrotranslate{page}\nobreakspace}%
            {\acrotranslate{pages}\nobreakspace}%
        \vspace{10pt}
        \tabularnewline
        }%
    }%
    \acroheading
    \acropreamble
    \noindent
    \begin{tabular}{@{}ll}
        \AcronymTable
    \end{tabular}
}

% Abkürzungsverzeichnis SETUP
\acsetup{
	list/heading = section*,
	list/name = {Abkürzungsverzeichnis},
    list/template = description,
	make-links = true,
	link-only-first = false
}

% Abkürzungsverzeichnis überschreibt \UseAcroTemplate für \ac
% New Counter to count used acronyms:
\newtotcounter{acro_num}
\def\oldUseAcroTemplate{} \let\oldUseAcroTemplate=\UseAcroTemplate
\def\UseAcroTemplate{\stepcounter{acro_num}\oldUseAcroTemplate}

% Abstände und Einrückungen abhängig von config.tex ein-/ausschalten
\if\CEINR 0
    \setlength{\parskip}{6pt}
    \setlength{\parindent}{0cm}
\fi
 aufgerufen werden, damit alle Template-Root-Dateien integriert werden
\makeatletter
\def\input@path{{../}{path1/}}
\makeatother

% die veraltete CAUTHOR Variable wird automatisch befüllt
\def\CAUTHOR{\CAUTHORVOR\ \CAUTHORNACH}

\newcounter{totalbibentries}
\newcommand*{\listcounted}{}

\makeatletter
\AtDataInput{
  \xifinlist{\abx@field@entrykey}\listcounted
    {}
    {\stepcounter{totalbibentries}
     \listxadd\listcounted{\abx@field@entrykey}}
}
\makeatother

% Standard Abkürzungsverzeichnis überschreiben -> einheitliche Einrückung
\RenewAcroTemplate[list]{description}{%
    \acronymsmapT{%
        \AcroAddRow{%
        \textbf{\acrowrite{short}}%
        &
        \acrowrite{long}%
        \acropages
            {\acrotranslate{page}\nobreakspace}%
            {\acrotranslate{pages}\nobreakspace}%
        \vspace{10pt}
        \tabularnewline
        }%
    }%
    \acroheading
    \acropreamble
    \noindent
    \begin{tabular}{@{}ll}
        \AcronymTable
    \end{tabular}
}

% Abkürzungsverzeichnis SETUP
\acsetup{
	list/heading = section*,
	list/name = {Abkürzungsverzeichnis},
    list/template = description,
	make-links = true,
	link-only-first = false
}

% Abkürzungsverzeichnis überschreibt \UseAcroTemplate für \ac
% New Counter to count used acronyms:
\newtotcounter{acro_num}
\def\oldUseAcroTemplate{} \let\oldUseAcroTemplate=\UseAcroTemplate
\def\UseAcroTemplate{\stepcounter{acro_num}\oldUseAcroTemplate}

% Abstände und Einrückungen abhängig von config.tex ein-/ausschalten
\if\CEINR 0
    \setlength{\parskip}{6pt}
    \setlength{\parindent}{0cm}
\fi

% ab hier ist das Arbeitsverzeichnis das root Verzeichnis (gilt primär für \input)

% Abkürzungen müssen früher eingefügt werden, da sie
% nicht als Text, sondern als Variablen definiert werden
% Definieren Sie hier Ihre Abkürzungen und Glossar-Einträge anhand der Beispiele.
% Wenn Sie diese dann im Text verwenden, rufen Sie einfach \gls{key} auf, z.B. \gls{ac:dhge}.
% LaTeX kümmert sich um den Rest.
% Wenn alle Abkürzungen auch ohne Verweis darauf generiert werden sollen, ist ein Schalter dafür in config.tex verfügbar.
% Eine ausführliche, anfängerfreundliche Dokumentation ist unter https://www.overleaf.com/learn/latex/Glossaries abrufbar.

%\newglossaryentry{gls:sso}{
%    name={Single-Sign-On},
%    description={Authentifizierung an mehreren Diensten mit einer Anmeldung}
%}

\newacronym[]{ac:dhge}{DHGE}{Duale Hochschule Gera-Eisenach}



% Konfiguration globaler Definitionen

% PDF Metadata
\hypersetup{
    pdftitle={\CTITLE},
    pdfauthor={\CAUTHOR}
}

\addbibresource	{../literatur.bib}

% Hier beginnt der Spaß
% setzt variablen für den Titel -> \maketitle in deckblatt.tex
\title{{\LARGE \textbf{\CTITLE}}}
\author{}
\date{}

\begin{document}
\pagenumbering{gobble}
    % lade Deckblatt ohne Nummerierung
    % DEFINITION SECTION

% legt den hSpace fest um die Einträge mittig zu platzieren
% 0.4375 berechnet sich aus:
% ((textwidth / 2) - (margin_left - margin_right)) / textwidth
% mit textwidth = pagewidth - margin_left - margin_right
% Dann muss nur noch parindent abgezogen werden
\def\defaultHSpace{\hspace{-\parindent}\hspace{0.4375\textwidth}}

\newcommand{\markBox}[2]
{
    \ifnum#1 = 1
        \def\checkboxes{#2 {$\boxtimes$} #2 {$\square$} #2 {$\square$} #2 {$\square$}}
    \else\ifnum#1 = 2
        \def\checkboxes{#2 {$\square$} #2 {$\boxtimes$} #2 {$\square$} #2 {$\square$}}
    \else\ifnum#1 = 3
        \def\checkboxes{#2 {$\square$} #2 {$\square$} #2 {$\boxtimes$} #2 {$\square$}}
    \else\ifnum#1 = 4
        \def\checkboxes{#2 {$\square$} #2 {$\square$} #2 {$\square$} #2 {$\boxtimes$}}
    \else
        \def\checkboxes{#2 {$\square$} #2 {$\square$} #2 {$\square$} #2 {$\square$}}
    \fi\fi\fi\fi

    \hspace*{-.5cm}\checkboxes
}

% Definition der Deckblatt-Einträge

\newcommand{\deckblattEntry}[2] {
    \begin{tabular}{rl}
        \defaultHSpace{} & \\ #1: & #2
    \end{tabular}
    % folgende newline ist notwendig damit die Formatierung angewendet wird

}


% DECKBLATT STRUKTUR SECTION
\vspace{\fill}
\maketitle

\if\CARBEIT B
    \begin{center}
        {\LARGE\bf Bachelorarbeit}

        \vspace{0.5cm}vorgelegt am \CDATUM
    \end{center}

    \vspace{1cm}
\else
    \begin{tabular}{rcccc}
        \defaultHSpace{} & I & II & III & IV \\
        {Projektarbeit Nr.}  \markBox{\CARBEIT}{&}
    \end{tabular}

    \deckblattEntry{vorgelegt am}{\CDATUM}
\fi

\deckblattEntry{von}{\CAUTHOR}
\deckblattEntry{Matrikelnummer}{\CMATRIKEL}
\deckblattEntry{DHGE Campus}{\CCAMPUS}
\deckblattEntry{Studienbereich}{\CBEREICH}
\deckblattEntry{Studiengang}{\CSTUDIENGANG}
\deckblattEntry{Kurs}{\CKURS}
\deckblattEntry{Ausbildungsstätte}{\CBETRIEB}
\deckblattEntry{\BETREUER}{\CBETREUER}

\vspace*{\fill}

\pagebreak


    % Lade Sperrvermerk abhängig von der Einstellung in config.tex
    \if\CSPERRVERMERK 1
        \def\CARBEITTYPNAME{Projektarbeit}

\if\CARBEIT B
    \def\CARBEITTYPNAME{Bachelorarbeit}
\fi

\vspace*{5.5cm}
\begin{center}
    {\LARGE\bf Sperrvermerk}

    \vspace*{1cm}
    Die vorgelegte \CARBEITTYPNAME{} mit dem Titel \enquote{\CTITLE{}} basiert auf internen, vertraulichen Daten und Informationen des Unternehmens \CBETRIEB{}.

    Diese \CARBEITTYPNAME{} darf nur vom Erst- und Zweitgutachter sowie berechtigten Mitgliedern des Prüfungsausschusses eingesehen werden.
    Eine Vervielfältigung und Veröffentlichung der \CARBEITTYPNAME{} ist auch auszugsweise nicht erlaubt.

    Die Vervielfältigung und Veröffentlichung der \CARBEITTYPNAME{} sowie die Einsichtnahme durch Dritte bedarf der ausdrücklichen
    Zustimmung des Verfassers und des Unternehmens.
\end{center}
\cleardoublepage

    \fi

    % Lade Thesenblatt abhängig von der Einstellung in config.tex
    \if\CARBEIT B
        % auskommentieren was nicht genutzt wird

\baFormat{Thesen zur Bachelorarbeit}
{
    \begin{enumerate}
        \item These 1
            \vspace{0.5cm}

        \item These 2
            \vspace{0.5cm}

        \item These 3
            \vspace{0.5cm}

        \item These 4
            \vspace{0.5cm}

        \item These 5
            \vspace{0.5cm}
    \end{enumerate}
}

\baFormat{Autorreferat zur Bachelorarbeit}{
    Beispieltext
}

    \fi

    % Lade Abstract abhängig von der Einstellung in config.tex
    \if\CHASABSTRACT 1
        \section*{Abstract}
        % hier können Sie Ihr Abstract schreiben

        \newpage % Sections haben hier noch keinen automatischen Seitenumbruch
    \fi

    % INHALTSVERZEICHNIS
    \pagenumbering{Roman} \setcounter{page}{1}
    \tableofcontents{\fancyfoot{}}
    \if\CKUSCHE 1 % im Kusche Mode erscheinen die Anlagen im InhaltsVZ
        \phantomsection
        \listofanlagen
    \fi
    \cleardoublepage

    % ABBILDUNGSVERZEICHNIS
    \phantomsection
    \conditionalLoF

    % TABELLENVERZEICHNIS
    \phantomsection
    \conditionalLoT

    \if\CKUSCHE 0
        \printnoidxglossary[title={Abkürzungsverzeichnis}]
        \cleardoublepage
    \fi

    \setcounter{savepage}{\arabic{page}}

    % MAIN CONTENT
    \pagenumbering{arabic}
    % hier können Sie Ihre Arbeit schreiben.
% für ein Beispiel siehe `build/tests/`
Piep

    \cleardoublepage

    % LITERATURVERZEICHNIS
    % TODO Formatierung

    % Im Kusche-Mode wird arabische Nummerierung beibehalten
    \if\CKUSCHE 0
        \pagenumbering{Roman} \setcounter{page}{\thesavepage}
    \fi

    % wenn keine Literatur verwendet (zitiert) wird, erstelle kein Literaturverzeichnis
    % im Kusche-Mode wird das LiteraturVZ später generiert
    \if\CKUSCHE 0
        \ifnum\thetotalbibentries>0
            \printbibliography[title=Literaturverzeichnis]
            \addcontentsline{toc}{section}{Literaturverzeichnis}
            \cleardoublepage
        \fi
    \fi


    % TODO %ANLAGENVERZEICHNIS UND ANLAGEN
    % erzeugt ein Anlagenkapitel und fügt es zum InhaltsVZ hinzu, insofern Anlagen existieren.

    % \listofanlagen generiert ein Anlagenverzeichnis
    % im Kusche-Modus wird kein Anlagenverzeichnis erzeugt, aber Anlagen im Inhaltsverzeichnis geführt
    \ifnum\totvalue{anlagenentries}>0
    \section*{Anlagen}
    \addcontentsline{toc}{section}{Anlagen}
    \if\CKUSCHE 0
        \begin{spacing}{2}
            \listofanlagen
        \end{spacing}
        \cleardoublepage
    \fi
    \fi

    \if\CKUSCHE 1
        % die Kapitel.lfd-Nummerierung wird im Kusche Mode ab hier wieder deaktiviert,
        % da Anhänge mit Buchstaben nummeriert werden sollen (TODO: das funktioniert noch nicht, help wanted)
        \counterwithout{figure}{section}
        \counterwithout{table}{section}

        % deaktiviere Kapitel in Fußzeile im Kusche Mode ab hier
        \fancyfoot      [L]{}
    \fi

    \renewFigTabCap % Verhalten von Figure und Table Environments sowie \caption wird verändert (TODO: wie genau? LG, ZPM)
    \begin{table}[H]
    \caption{Test Table}
    \begin{tabular}{| l | l |}
        Test & Test\\
        \hline Test & test
    \end{tabular}
\end{table}

\dhgefigure[h]{img}[scale=0.25]{Ein Testbild}{fig:anlagentest}[Xmisc][S. 17ff]
\dhgefigure{img}{Ein Testbild}{fig:anlagentest}[Xmisc]

\begin{figure}[H]
    \centering
    \caption{test}
    \includegraphics[scale=0.75]{img}
    \label{fig:anlagentest2}
\end{figure}

    \cleardoublepage

    \if\CKUSCHE 1
        % Im Kusche Mode kommt das Glossar erst hier unten
        \printnoidxglossary
        \cleardoublepage

        % Im Kusche-Mode kommt das LiteraturVZ zuletzt
        \ifnum\thetotalbibentries>0
            \printbibliography[title=Literaturverzeichnis]
            \addcontentsline{toc}{section}{Literaturverzeichnis}
            \cleardoublepage
        \fi
    \fi

    % Einrückung und Abstand unabhängig von config.tex setzen: Ehrenwort und Freigabe korrekt formatiert
    \setlength{\parskip}{6pt}
    \setlength{\parindent}{0cm}

    % EHRENWÖRTLICHE ERKLÄRUNG

    \pagestyle{empty}
    \pagenumbering{gobble}
    \section*{Ehrenwörtliche Erklärung}
    \addcontentsline{toc}{section}{Ehrenwörtliche Erklärung}
    Ich erkläre hiermit ehrenwörtlich,
\begin{flushleft}
\begin{enumerate}[leftmargin=0.5cm]
    \item 	dass ich meine {\if\CARBEIT BBachelorarbeit\else Projektarbeit\fi}
    mit dem Thema:  \\
    \vspace*{1cm}
            \textbf{\CTITLE} \\
    \vspace*{1cm}
            ohne fremde Hilfe angefertigt habe, \\
    \item	dass ich die Übernahme wörtlicher Zitate aus der Literatur sowie die Verwendung der
            Gedanken anderer Autoren an den entsprechenden Stellen innerhalb der Arbeit gekennzeichnet habe und  \\
    \item	dass ich meine {\if\CARBEIT BBachelorarbeit\else Projektarbeit\fi}
    bei keiner anderen Prüfung vorgelegt habe. \\
    \vspace*{1cm}
\end{enumerate}
\noindent
Ich bin mir bewusst, dass eine falsche Erklärung rechtliche Folgen haben wird.
\end{flushleft}
\vspace*{\fill}
\begin{tabular} {lrl}
    \hspace{5.5cm} & \hspace{3cm} & \hspace{5.5cm} \\
    \hrulefill & & \hrulefill \\
    Ort, Datum & & Unterschrift
\end{tabular}
\vspace*{\fill}

\cleardoublepage


    \if\CARBEIT B
        \begin{center}
    {\LARGE\bf Freigabeerklärung zur Bachelorarbeit}
\end{center}

\vspace{1cm}

\hspace*{-0.3cm}
\begin{tabular}{l l l}
    Name, Vorname Student{\ifnum\CAUTHORANR = 1 in\fi}: \CAUTHORNACH , \CAUTHORVOR&Matrikel-Nr.: \CMATRIKEL&Kurs: \CKURS\\
    \\
    Ausbildungsstätte: \CBETRIEB
\end{tabular}

\vspace{1.5cm}
Zur öffentlichen Einsichtnahme der Bachelorarbeit unsere{\ifnum\CAUTHORANR = 1 r\else s\fi} Student{\ifnum\CAUTHORANR = 1 in\else en\fi}, {\ifnum\CAUTHORANR = 1 Frau\else Herr\fi} \CAUTHOR , in der
Bibliothek der Dualen Hochschule Gera-Eisenach erklären wir uns:
\vspace{1cm}

\hspace{2cm}$\square$ einverstanden

\hspace{2cm}$\boxtimes$ nicht einverstanden.

\vspace{1.5cm}
{\bf Thema der Bachelorarbeit:}

\vspace{1cm}
\CTITLE
\vspace{1cm}

\vspace*{\fill}
\begin{tabular} {lrl}
    \hspace{5.5cm} & \hspace{1.5cm} & \hspace{5.5cm} \\
    \hrulefill     &              & \hrulefill     \\
    Datum, Stempel, Unterschrift Firma/Einrichtung& & Datum, Unterschrift Student{\ifnum\CAUTHORANR = 1 in\fi}
\end{tabular}
\vspace*{\fill}

    \fi

\end{document}
