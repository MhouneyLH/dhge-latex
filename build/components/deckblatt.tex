% DEFINITION SECTION

% legt den hSpace fest um die Einträge mittig zu platzieren
% 0.4375 berechnet sich aus:
% ((textwidth / 2) - (margin_left - margin_right)) / textwidth
% mit textwidth = pagewidth - margin_left - margin_right
% Dann muss nur noch parindent abgezogen werden
\def\defaultHSpace{\hspace{-\parindent}\hspace{0.4375\textwidth}}

\newcommand{\markBox}[2]
{
    \ifnum#1 = 1
        \def\checkboxes{#2 {$\boxtimes$} #2 {$\square$} #2 {$\square$} #2 {$\square$}}
    \else\ifnum#1 = 2
        \def\checkboxes{#2 {$\square$} #2 {$\boxtimes$} #2 {$\square$} #2 {$\square$}}
    \else\ifnum#1 = 3
        \def\checkboxes{#2 {$\square$} #2 {$\square$} #2 {$\boxtimes$} #2 {$\square$}}
    \else\ifnum#1 = 4
        \def\checkboxes{#2 {$\square$} #2 {$\square$} #2 {$\square$} #2 {$\boxtimes$}}
    \else
        \def\checkboxes{#2 {$\square$} #2 {$\square$} #2 {$\square$} #2 {$\square$}}
    \fi\fi\fi\fi

    \hspace*{-.5cm}\checkboxes
}

% Definition der Deckblatt-Einträge

\newcommand{\deckblattEntry}[2] {
    \begin{tabular}{rl}
        \defaultHSpace{} & \\ #1: & #2
    \end{tabular}
    % folgende newline ist notwendig damit die Formatierung angewendet wird

}


% DECKBLATT STRUKTUR SECTION
\vspace{\fill}
\maketitle

\if\CARBEIT B
    \begin{center}
        {\LARGE\bf Bachelorarbeit}

        \vspace{0.5cm}vorgelegt am \CDATUM
    \end{center}

    \vspace{1cm}
\else
    \begin{tabular}{rcccc}
        \defaultHSpace{} & I & II & III & IV \\
        {Projektarbeit Nr.}  \markBox{\CARBEIT}{&}
    \end{tabular}

    \deckblattEntry{vorgelegt am}{\CDATUM}
\fi

\deckblattEntry{von}{\CAUTHOR}
\deckblattEntry{Matrikelnummer}{\CMATRIKEL}
\deckblattEntry{DHGE Campus}{\CCAMPUS}
\deckblattEntry{Studienbereich}{\CBEREICH}
\deckblattEntry{Studiengang}{\CSTUDIENGANG}
\deckblattEntry{Kurs}{\CKURS}
\deckblattEntry{Ausbildungsstätte}{\CBETRIEB}
\deckblattEntry{\BETREUER}{\CBETREUER}

\vspace*{\fill}

\pagebreak
