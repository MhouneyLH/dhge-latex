\DeclareDocumentCommand{\dhgefigure}{O{tbp} m O{width=\textwidth, angle=0} m m O{} O{}}
{
    \begin{figure}[#1]
        \begin{center}
            \includegraphics[#3]{#2}
        \end{center}
        \caption{#4}
        \label{#5}

        \ifx #6\empty \else
            \ifx #7\empty
                {\small \protect \textbf{Quelle:} \cite{#6}}
                \else
                {\small \protect \textbf{Quelle:} \cite[#7]{#6}}
            \fi
        \fi

    \end{figure}
}

\DeclareDocumentCommand{\dhgesvgfigure}{O{tbp} m O{width=\textwidth, angle=0} m m O{} O{}}
{
    \begin{figure}[#1]
        \begin{center}
            \includesvg[#3]{#2}
        \end{center}
        \caption{#4}
        \label{#5}

        \ifx #6\empty \else
            \ifx #7\empty
                {\small \protect \textbf{Quelle:} \cite{#6}}
                \else
                {\small \protect \textbf{Quelle:} \cite[#7]{#6}}
            \fi
        \fi

    \end{figure}
}

% SubSubSubSection
\newcommand{\dhgeparagraph}[1]{\paragraph{#1}\mbox{}\\}

% Doppelte Unterstreichung
\newcommand{\doubleunderline}[1]{
    \underline{\underline{#1}}
}

% Formatierung der Bachelorarbeit: Autorreferat und Thesenblatt
\newcommand{\baFormat}[2]{
    \begin{center}
        {\LARGE\bf #1}

        \vspace{0.7cm}
        {\large\bf\enquote{\CTITLE}}

        \vspace{0.5cm}
        von \CAUTHOR
    \end{center}

    \vspace{1.5cm}

    {#2}

    \cleardoublepage
}
