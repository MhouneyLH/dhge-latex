% Projektarbeit Nr. (1 bis 4) oder Bachelorarbeit (B)
\def\CARBEIT        {1}

% Title der Arbeit
\def\CTITLE         {THEMA}

% Author der Arbeit (ANR -> Anrede, VOR -> Vorname, NACH -> Nachname)
\def\CAUTHORANR     {0}     % 1 -> Frau !1 -> Herr
\def\CAUTHORVOR     {VOR}
\def\CAUTHORNACH    {NACH}

% "vorlege am" - Datum
\def\CDATUM         {\today}

% Matrikelnummer des Authors
\def\CMATRIKEL      {MTR-NR}

% Kurs des Authors
\def\CKURS          {KURS}

% DHGE Campus des Authors (Gera/Eisenach)
\def\CCAMPUS        {Gera}

% Studienbereich des Authors
\def\CBEREICH       {Technik}

% Studiengang des Authors
\def\CSTUDIENGANG   {STUDIENGANG}

% Betrieb des Authors (nur der Name des Betriebs keine Adresse)
\def\CBETRIEB       {FIRMA}

% Betreuer der Arbeit (den akademischen Titel nicht vergessen)
\def\CBETREUER      {BETREUER}

% Fügt einen einfachen Sperrvermerk hinter das Deckblatt (1 = aktiv)
\def\CSPERRVERMERK  {0}

% längste Abkürzung in der abk.tex
\def\CABKL          {DHGE}

% verwende Richtlinien nach Prof. Dr. Kusche
% zum Aktivieren auf 1 setzen
\def\CKUSCHE        {0}

% 1 aktiviert das Einbinden des Abstracts
\def\CHASABSTRACT   {0}

% 1 aktiviert das Glossar (Abkürzungsverzeichnis)
\def\CHASGLO        {1}

% 1 aktiviert den Font-Vorschlag, 0 deaktiviert ihn
\def\CFANCYFONTS    {1}

% 1 setzt Absatztrenner auf Einrückungen, 0 auf vertikale Abstände
\def\CEINR          {0}

% 1 lässt alle Glossar- und Abkürzungs-Einträge unbedingt erzeugen, 0 nur bei Referenz darauf
\def\CALLGLO        {1}
